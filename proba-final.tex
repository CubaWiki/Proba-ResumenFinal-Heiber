\documentclass[a4paper,spanish]{article}

\usepackage[spanish,activeacute]{babel}
\usepackage{moreverb}
\usepackage{fancyhdr}
\usepackage{graphicx}
\usepackage{multicol}
\usepackage{theorem,amsmath,amssymb,latexsym}
\usepackage{enumerate}

\oddsidemargin -0.5in
\textwidth 7.2in
\topmargin 0in
\addtolength{\topmargin}{-.5in}
\textheight 10in
\parskip=1ex
\pagestyle{fancy}
%usar el segundo nivel de enumeracion con letras
%\Rnewcommand{\labelenumii}{\alph{enumii}. }

\newcommand{\Rv}{\marginpar{REVISAR}}
\newcommand{\nohecho}{\marginpar{NO HECHO}}

%espaciado
\newcommand{\vsp}{\vspace{0.4cm}}
\newcommand{\hsp}{\hspace*{0.12cm}}

%comandos para el resumen
\newcommand{\tab}{\hspace*{1cm}}
\newcommand{\llamada}[1]{\begin{center} \bfseries #1 \mdseries \end{center}}
\newcommand{\nota}[1]{\vsp\defi{Nota}{#1}\vsp}
\newcommand{\R}[0]{\mathbb{R}}
\newcommand{\N}[0]{\mathbb{N}}
\newcommand{\norma}[1]{\left\|#1\right\|}
\newcommand{\limite}[2]{\lim_{ #1 \rightarrow #2}}
\newcommand{\xx}[0]{\mathbf{x}}
\newcommand{\xO}[0]{\mathbf{x_0}}
\newcommand{\yO}[0]{\mathbf{y_0}}
\newcommand{\comp}[0]{\circ}
\newcommand{\parcial}[2]{\frac{\partial #1}{\partial #2}}
\newcommand{\D}[0]{\mathbf{D}}
\newcommand{\He}[0]{\mathbf{H}}
%\newcommand{\J}[0]{\mathbf{J}}
\newcommand{\grad}[0]{\bigtriangledown}
\newcommand{\eps}[0]{\varepsilon}
\newcommand{\lthen}[0]{\Rightarrow}
\newcommand{\piso}[1]{\left\lfloor{x}\right\rfloor}
\newcommand{\bigfrac}[2]{\displaystyle\frac{#1}{#2}}
\newcommand{\bigchoose}[0]{\displaystyle\choose}
\DeclareMathOperator{\Cov}{Cov}
\newcommand{\expon}[0]{{\cal E}}
\newcommand{\tiendep}[0]{\longrightarrow^p}
\newcommand{\tiended}[0]{\longrightarrow^d}
\DeclareMathOperator{\MAD}{MAD}
\DeclareMathOperator{\ECM}{ECM}

\newtheorem{teo}{Teorema}
\newtheorem{lema}{Lema}
\newtheorem{coro}{Corolario}
\newtheorem{defi}{Definici\'on}
\newtheorem{obs}{Observaci\'on}

% proof
\newenvironment{demo}{{\noindent \textbf{Demo: }}}{\hfill\rule{2mm}{2mm}\par}

\lhead{Probabilidad y Estad\'istica (C)}
\rhead{Apunte de repaso general}

\cfoot{$\thepage$ de \pageref{theend}}

\begin{document}

Disclaimer: Este apunte no es autocontenido y fue pensado como un repaso 
de los conceptos, no para aprenderlos de aqu'i directamente.

\section{Teor'ia de probabilidad}

\begin{defi}[espacio muestral]
Un \emph{espacio muestral} es el conjunto de resultados posibles de un
experimento.
\end{defi}

\begin{defi}[evento]
Un \emph{evento} o \emph{suceso} es un subconjunto del espacio muestral.
\end{defi}

\begin{defi}[frecuencia relativa y probabilidad]
La \emph{frecuencia relativa} del evento $A$ est'a dada por la cantidad de 
veces que ocurre $A$ $n_A$ sobre la cantidad de veces que se hace el 
experimento $n$. $fr(A) = n_A/ n$. Cuando $n$ tiende a infinito $fr(A)$ tiende
a $P(A)$, la probabilidad del evento $A$. Tanto la frecuencia relativa como la
probabilidad estan entre $0$ y $1$.
\end{defi}

\begin{defi}[teor'ia axiom'atica de probabilidad]
La teor'ia axiom'atica de probabilidad est'a definida por los siguientes
axiomas (probabilidades sobre el espacio muestral $S$):
\begin{enumerate}
\item $P(A) \geq 0$.
\item $P(S) = 1$.
\item Si $\forall i \neq j A_i \cap A_j = \emptyset$ entonces 
$P(\bigcup_i A_i) = \sum_i P(A_i)$.
\end{enumerate}
\end{defi}

\begin{teo}[propiedades de la probabilidad]
La funci'on de probabilidad $P$ cumple lo siguiente:
\begin{enumerate}
\item $P(A^c) = 1 - P(A)$.
\item $P(\emptyset) = 0$.
\item Si $A \subseteq B$ entonces $P(A) \leq P(B)$ y 
	$P(B \setminus A) = P(B) - P(A)$.
\item $P(A \cup B) = P(A) + P(B) - P(A \cap B)$.
\end{enumerate}
\end{teo}

\begin{defi}[espacio de equiprobabilidad]
Un espacio de equiprobabilidad $S$ es tal que $\forall x \in S$ se cumple
$P({x}) = 1 / \#S$.
\end{defi}

\begin{defi}[probabilidad condicional]
La probabilidad condicional de $A$ dado que sucedi'o $B$ es la probabilidad de
que suceda el evento $A$ con el espacio muestra reducido a $B$. Escribimos
$$P(A | B) = \frac{P (A \cap B)}{P(B)}.$$
\end{defi}

\begin{teo}[propiedades de la probabilidad condicional]
Las probabilidades condicionales sobre el espacio muestral $S$ cumplen:
\begin{enumerate}
\item $P(A|B) \geq 0$
\item $P(S|B) = 1$
\end{enumerate}
\end{teo}

\begin{teo}[regla del producto]
Si $P(A)>0$ y $P(B)>0$ entonces $P(A \cap B) = P(A|B)P(B) = P(B|A)P(A)$.
\end{teo}

\begin{teo}[regla del producto generalizada]
Si $\forall k P(A_1 \cap ... \cap A_k) > 0$ entonces
$$P(A_1 \cap ... \cap A_n) = P(A_1)P(A_2|A_1)P(A_3|A_1 \cap A_2) ...
	P(A_n|A_1 \cap ... \cap A_{n-1}).$$
\end{teo}

\begin{defi}[partici'on]
$A_1,...,A_k$ se dice una \emph{partici'on del espacio muestral} $S$ sii 
$\forall i,j$:
\begin{enumerate}
\item $A_i \cap A_j = \emptyset$
\item $P(A_i) > 0$
\item $\bigcup_{i=1}^k A_i = S$.
\end{enumerate}
\end{defi}

\begin{teo}[teorema de la probabilidad total]
Si $A_1,...,A_k$ es una partici'on de $S$ y $B$ un evento, entonces
$$P(B) = \sum_{i=1}^k P(B|A_i)P(A_i).$$
\end{teo}

\begin{teo}[teorema de Bayes]
Si $A_1,...,A_k$ es una partici'on de $S$ y $B$ un evento, entonces
$$P(A_j | B) = \frac{P(B | A_j)P(A_j)}{\sum_{i=1}^k P(B|A_i)P(A_i)}.$$
\end{teo}

\begin{defi}[independencia]
Los eventos $A$ y $B$ son independientes sii $P(A \cap B) = P(A)P(B)$. 
Equivalentemente $P(A) = P(A|B)$ y $P(B)=P(B|A)$.
\end{defi}

\begin{teo}[independencia del complemento]
Si $A$ y $B$ son independientes, $A$ y $B^c$ tambi'en lo son.
\end{teo}

\begin{defi}[independencia m'ultiple]
Un conjunto de eventos es independiente si para cualquier subconjunto de ellos
la probabilidad de la intersecci'on es el producto de las probabilidades.
\end{defi}

\begin{obs}
Si un conjunto es independiente, sus elementos son independientes de a pares,
pero no al rev'es.
\end{obs}

\section{Variables Aleatorias Discretas}

\begin{defi}[variable aleatoria discreta]
Una \emph{variable aleatoria (v.a.) discreta} es una funci'on
$X : S \to \R$ del espacio muestral en los reales. Puede verse como 
una asignaci'on de probabilidades a distintos valores.
\end{defi}

\begin{defi}[funci'on de probabilidad puntual]
La \emph{funci'on de probabilidad puntual} $p_X$ de la v.a. $X$ es la funci'on
que dado un real dice la probabilidad de que $X$ valga eso, es decir
$$p_X(x) = P(X = x) = P(\{w \in S | X(w) = x\}).$$
\end{defi}

\begin{teo}[propiedades de la probabilidad puntual]
La funci'on de distribuci'on acumulada $p_X$ de una v.a. $X$ cumple:
\begin{enumerate}
\item $\forall x p_X(x) \geq 0$
\item $\sum_{x \in Im(X)} p_X(x) = 1$
\end{enumerate}
\end{teo}

\begin{defi}[funci'on de distribuci'on acumulada]
La \emph{funci'on de distribuci'on acumulada} de una v.a. discreta $X$ $F_X$
se define para todo $x$ en $\R$ como
$$F_X(x) = P(X \leq x) = \sum_{y \leq x, y \in Im(X)} p_X(y).$$
\end{defi}

\begin{teo}[propiedades de la distribuci'on acumulada]
La funci'on de distribuci'on acumulada $F_X$ de una v.a. $X$ cumple:
\begin{enumerate}
\item $\forall x \in \R\ F_X(x) \in [0,1]$
\item $F_X$ es mon'otona no decreciente
\item $F_x$ es continua a derecha
\item $\limite{x}{\infty} F_X(x) = 1$ y $\limite{x}{-\infty} F_X(x) = 0$
\item En cada punto, el valor del salto es la probabilidad puntual: 
	$p_X(x) = F_X(x) - F_X(x^-)$.
\end{enumerate}
\end{teo}

\begin{teo}[probabilidad de un intervalo]
Sean $a,b \in \R, a \leq b$. Se cumple que:
\begin{enumerate}
\item $P(a < X \leq b) = F_x(b) - F_x(a)$
\item $P(a < X < b) = F_x(b^-) - F_x(a)$
\item $P(a \leq X \leq b) = F_x(b) - F_x(a^-)$
\item $P(a \leq X < b) = F_x(b^-) - F_x(a^-)$
\end{enumerate}
\end{teo}

\begin{defi}[esperanza]
La \emph{esperanza} de una v.a. discreta $X$ se define como 
$$E(X) = \mu_X = \sum_{x \in Im(X)} x p_X(x).$$
\end{defi}

\begin{teo}[funci'on de una v.a. discreta]
Si $X$ es una v.a. discreta entonces $f(X)$ tambi'en lo es y su esperanza es
$$E(f(x)) = \sum_{y \in Im(f(X))} y p_{f(X)}(y) = 
	\sum_{x \in Im(X)} f(x) p_X(x).$$
\end{teo}

\begin{teo}[linealidad de la esperanza]
Si $a,b \in \R$, $E(aX+b) = aE(X)+b$.
\end{teo}

\begin{defi}[varianza y desv'io]
La \emph{varianza} de una v.a. discreta $X$ se define como
$$V(X) = \sigma^2_X = \sum_{x \in Im(X)} (x - \mu_X)^2 p_X(X) = 
	E[(X-E(X))^2].$$
El \emph{desv'io standard} se define como $\sigma_X = \sqrt{\sigma^2_X}$.
\end{defi}

\begin{teo}[varianza]
$V(X) = E(X^2) - E(X)^2$.
\end{teo}

\begin{teo}[la varianza es bilineal]
$V(aX+b) = a^2 V(X)$ ($\sigma_{aX+b} = |a|\sigma_X$).
\end{teo}

\subsection{Distribuci'on Binomial}

Se realizan $n$ experimentos independientes con probabilidad de 'exito $p$. Si
$n=1$ es llama distribuci'on de Bernoulli. La variable binomial es la v.a. 
$X$:n'umero de 'exitos. Escribimos 
$$X \sim Bi(n,p)$$.

\begin{teo}[probabilidad puntual de distribuci'on binomial]
La funci'on de probabilidad puntual de $X \sim Bi(n,p)$ es 
$$p_X(k) = 
	\begin{cases}
		{n \choose k}p^k(1-p)^{n-k} & k \in \N, 0 \leq k \leq n \\
		0 & \mbox{otherwise}
	\end{cases}
$$
\end{teo}

\begin{obs}
Por la f'ormula de binomio de Newton $(a+b)^n = 
	\sum_{k=0}^n {n \choose k} a^k b^{n-k}$ podemos ver que 
$\sum_x p_X(x) = (p + (1-p))^n = 1$.
\end{obs}

\begin{teo}[esperanza y varianza de distribuci'on binomial]
Si $X \sim Bi(n,p)$ entonces se cumple que $E(X) = np$ y $V(X) = np(1-p)$.
\end{teo}

\begin{teo}[l'imite de la distribuci'on binomial]
Supongamos que $n \to \infty$ y $p \to 0$ tal que 
$np = \lambda$, y sea $X \sim Bi(n,p)$ entonces:
$$p_X(k) = {n \choose k}p^k(1-p)^{n-k} \longrightarrow 
	\frac{e^{-\lambda} \lambda^k}{k!}.$$
O sea que podemos aproximar, cuando $n$ es grande y $p$ chico, la distribuci'on
por su l'imite. Ver distribuci'on de Poisson.
\end{teo}

\subsection{Distribuci\'on Geom\'etrica}

La v.a. $X$ tiene distribuci'on geom'etrica $G(p)$ si se hacen ensayos
independientes con probabilidad de 'exito $p$ hasta obtener un 'exito
y se cuenta cuantos se hicieron.

\begin{teo}[probabilidad puntal de la geom'etrica]
Si $X \sim G(p)$ entonces $p_X(k) = (1-p)^{k-1}p \forall k \in \N$.
\end{teo}

\begin{teo}[distribuci'on acumulada de la geom'etrica]
Si $X \sim G(p)$ entonces $F_X(x) = 1-(1-p)^{\piso{x}}$.
\end{teo}

\begin{teo}[esperanza y varianza de la geom'etrica]
Si $X \sim G(p)$ entonces
$$E(X) = \frac{1}{p} \mbox{\ \ y\ \ } V(X)=\frac{1-p}{p^2}.$$
\end{teo}

\begin{teo}[propiedad de falta de memoria]
Si $X \sim G(p)$ y $n,m \in \N$ entonces $P(X > n+m |X > n) = P(X > m)$.
\end{teo}

\subsection{Distribuci\'on Binomial Negativa}

La v.a. $X$ tiene distribuci'on binomial negativa $BN(r,p)$ si se
hacen ensayos independientes con probabilidad de 'exito $p$ hasta
obtener $r$ 'exitos ($r geq 1$). Generaliza la geom'etrica.

\begin{teo}[probabilidad puntal de la binomial negativa]
Si $X \sim BN(r,p)$ entonces
$$p_X(k) = {k-1 \choose r-1} (1-p)^{k-r}p^r \forall k \in
\N_{geq r}.$$
\end{teo}

\begin{teo}[esperanza y varianza de la binomial negativa]
Si $X \sim BN(r,p)$ entonces
$$E(X) = \frac{r}{p} \mbox{\ \ y\ \ } V(X)=\frac{r(1-p)}{p^2}.$$
\end{teo}

\subsection{Distribuci\'on Hipergeom'etrica}

Se tienen $N$ individuos, con $D$ 'exitos y $N-D$ fracasos. Se extrae una
muestra de $n \leq N$. Sea la variable $X$ la cantidad de 'exitos en la
muestra, enonces $X$ tiene distribuci'on hipergeom'etrica $X \sim H(n,N,D)$.

\begin{teo}[probabilidad puntal de la hipergeom'etrica]
Si $X \sim H(n,N,D)$ entonces
$$P_X(k) = \frac{ {D \choose k} {N-D \choose n-k} }
                { {N \choose n}  } \hspace{1cm} 
                	max(0,n-(N-D)) \leq k \leq min(n,D).$$
\end{teo}

\begin{teo}[esperanza y varianza de la hipergeom'etrica]
Si $X \sim H(n,N,D)$ entonces
$$E(X) = n\frac{D}{N} \mbox{\ \ y\ \ } V(X)=\left(\frac{N-n}{N-1}\right)
	n\frac{D}{N}\left(1-\frac{D}{N}\right).$$
\end{teo}

El t'ermino $\frac{N-n}{N-1}$ se llama factor de correcci'on por poblaci'on
finita. Si $n << N$, el factor tiende a $1$ y se puede aproximar la 
hipergeom'etrica por una binomial $H(n,N,D) \sim Bi(n,N/D)$.

\subsection{Distribuci'on de Poisson}

Si $X \sim P(\lambda)$ tiene distribuci'on de poisson su funci'on de
probabilidad puntual est'a dada por
$$p_X(k) = \frac{e^{-\lambda} \lambda^k}{k!}.$$


\begin{teo}[esperanza y varianza de la poisson]
Si $X \sim P(\lambda)$ entonces
$$E(X) = \lambda \mbox{\ \ y\ \ } V(X)=\lambda.$$
\end{teo}

Un proceso de poisson es lo que pasa en un intervalo de tiempo dividible en
subintervalos tal que la probabilidad de la ocurrencia de un evento en cada
subintervalo es independiente de los otros y que la probabilidad de ocurrencia
de mas de un evento en el mismo subintervalo es despreciable. Esto hace que
se aproximable por una binomial tal que $n \to \infty$ y 
$p \to 0$ si tendemos el subintervalo a $0$. Esto hace que la
distribuci'on de la cantidad de eventos en un intervalo grande de largo $t$ es
de Poisson con par'ametro $\lambda = np$, porque su funci'on de probabilidad
puntual es el l'imite de la binomial.

\section{Variables Aleatorias Continuas}

Las v.a. continuas resultan de tomar limite de discretizaci'on en una variable
discreta. Formalmente:
\begin{defi}[variable aleatoria continua]
$X$ es una \emph{variable aleatoria continua} si existe una funci'on 
$f_X : \R \to \R_{\geq 0}$ llamada \emph{funci'on de densidad} tal que
$$\forall A \subseteq \R, P(X \in A) = \int_A f_X(x)dx.$$
\end{defi}

\begin{obs}
Observamos que las funciones de densidad difieren de las de probabilidad 
puntual:
\begin{enumerate}
\item $\forall a P(X = a) = \int_{\{a\}} f_X(x) dx = 0$
\item $f(a)$ no es una probabilidad, de hecho puede ser mayor a $1$.
\end{enumerate}
\end{obs}

\begin{defi}[funci'on de distribuci'on acumulada para continuas]
La \emph{funci'on de distribuci'on acumulada} de una v.a. continua $X$ se
define como la probabilidad del intervalo $(-\infty, x)$ y puede escribirse
como 
$$F_X(x) = \int_{-\infty}^x f_X(t) dt.$$
\end{defi}

\begin{teo}[propiedades de la acumulada]
Las funciones de distribuci'on acumulada de v.a. continuas cumplen:
\begin{enumerate}
\item $\forall x \in \R F_X(x) \in [0,1]$
\item $F_X$ es mon'otona no decreciente
\item \label{F-continua}$F_X$ es continua en todo punto.
\item $\limite{x}{\infty} F_X(x) = 1$ y $\limite{x}{-\infty} F_X(x) = 0$.
\end{enumerate}
\end{teo}
Notar que todos salvo \ref{F-continua} son iguales a las propiedades para 
discretas. La propiedad \ref{F-continua} vale porque la discontinuidad a 
izquierda se pierde porque al continuizar la variable se suaviza la curva.

\begin{teo}[funciones de variables aleatorias]
Si $X$ es una v.a. y $g : \R \to \R$ mon'otona creciente y biyectiva entonces 
$F_{g(X)} = F_X \comp g^{-1}$ y $f_{g(X)} = (f_X \comp g^{-1}) {g^{-1}}'$.
\end{teo}

\begin{defi}[percentiles]
Si $X$ es una v.a. continua el percentil $(100 p)$-'esimo de la distribuci'on
de $X$ es $x_p$ tal que $F_X(x_p) = p$.
\end{defi}

\begin{defi}[mediana y cuartiles]
La mediana $\tilde{\mu}$ se define como el percentil $50$-'esimo de la
distribuci'on. Los cuartiles se definen como los percentiles $25$ y 
$75$-'esimos.
\end{defi}

\begin{defi}[esperanza de continuas]
Sea $X$ una v.a. continua. La esperanza de $X$ se define como
$$E(X) = \mu_X = \int_{-\infty}^\infty x f_X(x) dx.$$
\end{defi}

\begin{teo}[funciones de una v.a. continua]
Si $X$ es una v.a. continua, $f(X)$ tambi'en lo es y
$$E(f(X)) = \int_{-\infty}^\infty f(x) f_X(x) dx.$$
\end{teo}

\begin{teo}[linealidad de la esperanza]
Si $a,b \in \R$, $E(aX+b) = aE(X)+b$.
\end{teo}

\begin{defi}[varianza y desv'io de continuas]
Sea $X$ una v.a. continua. La varianza de $X$ se define como
$$V(X) = \sigma^2_X = E[(X-\mu_X)^2] = 
	\int_{-\infty}^\infty (x - \mu_x)^2 f_X(x) dx$$
y el desv'io $\sigma_X = \sqrt{\sigma^2_X}$.
\end{defi}

\begin{teo}[varianza]
$V(X) = E(X^2) - E(X)^2$.
\end{teo}

\begin{teo}[la varianza es bilineal]
$V(aX+b) = a^2 V(X)$ ($\sigma_{aX+b} = |a|\sigma_X$).
\end{teo}

\begin{defi}[funci'on indicadora]
La funci'on indicadora del conjunto $A$ $I_A : \R \to \R$ se define como:
$$I_A(x) = \begin{cases} 1 & x \in A \\ 0 & x \notin A. \end{cases}$$
\end{defi}

\subsection{Distribuci'on Uniforme}

La distribuci'on uniforme consiste en elegir un n'umero al azar en un 
intervalo $[A,B]$ de manera que cualquier subintervalo tiene una probabilidad
proporcional a su largo.

\begin{teo}[densidad de la uniforme]
Si $X \sim U(A,B)$ ($X$ tiene distribuci'on uniforme) entonces
$$f_X(x) = 1/(B-A).$$
\end{teo}

\begin{teo}[funci'on de distribuci'on acumulada de la uniforme]
Si $X \sim U(A,B)$ entonces
$$F_X(x) = \frac{x-A}{B-A} I_{[A,B]}(x) + I_{(B,\infty)}(x).$$
\end{teo}

\begin{teo}[esperanza y varianza de la uniforme]
Si $X \sim U(A,B)$ entonces 
$$E(X) = \frac{A+B}{2} \mbox{\ \ y\ \ } V(X) = \frac{(B-A)^2}{2}.$$
\end{teo}

\subsection{Distribuci'on Normal}

$X$ tiene \emph{distribuci'on normal} ($X \sim N(\mu,\sigma^2)$) sii su 
funci'on de densidad es
$$\frac{1}{\sqrt{2 \pi} \sigma} 
	e ^ {\displaystyle -\frac{(x-\mu)^2}{2 \sigma^2}}.$$
El gr'afico de dicha funci'on es una campana con eje de simetr'ia y m'aximo en
$x = \mu$ y puntos de inflexi'on en $x = \mu-\sigma$ y $x = \mu+\sigma$.

\begin{defi}[normal standard]
Definimos la distribuci'on \emph{normal standard} como $Z \sim N(0,1)$.
\end{defi}

\begin{teo}[estandarizaci'on de normales]
$$X \sim N(\mu,\sigma^2) \Leftrightarrow \frac{X-\mu}{\sigma} \sim N(0,1).$$
\end{teo}

\begin{teo}[acumulada de la normal]
La funci'on acumulada de la normal no puede escribirse como algo que no sea
una integral. Si $X \sim Z$ entonces $F_X(x) = \Phi(x)$.
\end{teo}
Dado el anterior teorema, la acumulada de la normal standard se encuentra
tabulada. De dicha tabla, aplicando la standarizaci'on, se puede obtener
la acumulada de cualquier normal.

\begin{teo}[esperanza y varianza de la normal]
Si $X ~ N(\mu,\sigma^2)$ entonces $E(X) = \mu$ y $V(X) = \sigma^2$.
\end{teo}

\subsection{Distribuci'on Gamma}

\begin{defi}[funci'on Gamma o factorial]
$$\Gamma(\alpha) = \int_0^\infty x^{\alpha-1} e^{-x}.$$
\end{defi}

\begin{teo}[propiedades de la funci'on Gamma]
\begin{enumerate}
\item Si $\alpha > 1$ entonces $\Gamma(\alpha) = (\alpha-1)\Gamma(\alpha-1)$.
\item Si $\alpha \in \N$ entonces $\Gamma(\alpha) = (\alpha-1)!$.
\item $\Gamma(1/2) = \sqrt{\pi}$.
\end{enumerate}
\end{teo}

Se dice que $X$ tiene distribuci'on Gamma ($X \sim \Gamma(\alpha,\lambda)$) si
su funci'on de densidad es
$$f_X(x) = 
	\frac{e^{-\lambda x} x^{\alpha-1} \lambda^\alpha}{\Gamma(\alpha)}.$$
Si $\lambda = 1$ la distribuci'on es Gamma standard de par'ametro $\alpha$.
La distribuci'on de la Gamma standard est'a tabulada para distintos valores de
$\alpha$.

\begin{teo}[esperanza y varianza de la Gamma]
Si $X \sim \Gamma(\alpha,\lambda)$ entonces $E(X) = \alpha/\lambda$ y 
$V(X) = \alpha/\lambda^2$.
\end{teo}

\begin{teo}[standarizaci'on de la Gamma]
Si $\lambda > 0$ entonces
$$X \sim \Gamma(\alpha,\lambda) \Leftrightarrow 
	\lambda X \sim \Gamma(\alpha,1).$$
\end{teo}

\subsection{Distribuci'on Exponencial}

Es un caso particular de la Gamma con $\alpha = 1$. Es decir, $X$ tiene 
distribuci'on exponencial ($X \sim \expon(\lambda)$) si su funci'on de
densidad es
$$f_X(x) = \lambda e^{-\lambda x} I_{(0,\infty)}(x).$$

\begin{teo}[acumulada de la exponencial]
Si $X \sim \expon(\lambda)$ entonces
$$F_X(x) = 1-e^{-\lambda x} I_{(0,\infty)}.$$
\end{teo}

\begin{teo}[esperanza y varianza de la exponencial]
Si $X \sim \expon(\lambda)$ entonces $E(X) = \lambda^{-1}$ y 
$V(X) = \lambda^{-2}$.
\end{teo}

\begin{teo}[falta de memoria]
Si $X \sim \expon(\lambda)$ y $s,t \in \R_{>0}$ entonces
$$P(X > s+t | X > s) = P(X > t).$$
\end{teo}

\begin{teo}[relaci'on entre distribuci'on exponencial y Poisson]
Sea un proceso de Poisson de tasa media $v$ por lo cual la variable $X_t$:
cantidad de eventos en un intervalo $t$ es $X_t \sim P(v t)$ entonces
la v.a. $X$: tiempo hasta el primer evento tiene distribuci'on exponencial
$X \sim \expon(v)$.
\end{teo}

\section{Momentos}

\begin{defi}[momento]
El \emph{momento de orden $k$} de la v.a. $X$ se define como $E(X^k)$.
\end{defi}

\begin{obs}[momentos 1 y 2]
El momento de orden $1$ $E(X) = \mu$ y el momento de orden $2$ 
$E(X^2) = \mu^2 + \sigma^2$.
\end{obs}

\begin{defi}[funci'on generadora de momentos]
LA \emph{funci'on generadora de momentos} de la v.a. $X$ se define como
$M_X(t) = E(e^tx)$.
\end{defi}

\begin{teo}[funci'on generadora de momentos]
$E(X^n) = \parcial{^n}{^n t} M_X(0)$.
\end{teo}

Las funciones generadoras de momentos de las distintas distribuciones
mencionadas puede encontrarse en la secci'on~\ref{sec:resumen}.

\begin{teo}[unicidad de la generadora de momentos]
Dada una distribuci'on la funci'on generadora de momentos existe y es 'unica.
Dada una funci'on generadora de momentos, la distribuci'on es 'unica.
\end{teo}
Dado el teorema anterior, la funci'on generadora de momentos sirve para 
definir completamente la distribuci'on. Esto es 'util para demostraciones
de que tal v.a. tiene tal distribuci'on.

\section{Generaci'on de n'umeros al azar}

Para generar n'umeros al azar, la idea es generar $X \sim U(0,1)$ y a partir 
de ella generar otras distribuciones como $F(X)$.

\begin{teo}Sea $U \sim U(0,1)$ y $G$ una acumulada continua y estrictamente
creciente. Si $X = G^{-1}(U)$ entonces $F_X = G$.
\end{teo}

\begin{teo} Sea $U \sim U(0,1)$ y $G$ una acumulada. Existe $H$ tal que 
$H(U)$ tiene funci'on de distribuci'on acumulada $G$.
\end{teo}

\section{Vectores aleatorios}

\newcommand{\pxy}[0]{p_{XY}}
\newcommand{\fxy}[0]{f_{XY}}
\newcommand{\Fxy}[0]{F_{XY}}

\begin{defi}[probabilidad conjunta]
Sean $X$ e $Y$ v.a. discretas sobre el mismo espacio muestral $S$. La 
\emph{funci'on de probabilidad conjunta} del par $(X,Y)$ se define como:
$\pxy(x,y) = P(X = x, Y = y)$.
\end{defi}

\begin{teo}[propiedades de la probabilidad conjunta]
Una funci'on de probabilidad conjunta satisface
\begin{enumerate}
\item $\forall x,y \pxy(x,y) \geq 0$
\item $\sum_x \sum_y \pxy(x,y) = 1$.
\end{enumerate}
\end{teo}

\begin{defi}[probabilidad marginal]
Las \emph{funciones de probabilidad marginal} de $X$ e $Y$ est'an dadas por:
$$p_X(x) = \sum_y \pxy(x,y) \mbox{\ \ y\ \ } p_Y(y) = \sum_x \pxy(x,y).$$
\end{defi}

\begin{defi}[acumulada conjunta discreta] La \emph{funci'on de distribuci'on
acumulada conjunta} de $(X,Y)$ est'a dada por
$$\Fxy(x,y) = \sum_{s \leq x} \sum_{t \leq y} \pxy(s,t).$$
\end{defi}

\begin{defi}[densidad conjunta] $(X,Y)$ es un vector aleatorio continuo
si existe una funci'on llamada \emph{funci'on de densidad conjunta} $\fxy$ tal
que:
$$P((X,Y) \in A \subseteq \R^2) = \iint_A \fxy(x,y)\ dx\ dy.$$
\end{defi}

\begin{obs} En particular si $A = [a,b] \times [c,d]$, 
$$P(x \in A) = \int_a^b \int_c^d \fxy(x,y)\ dx\ dy.$$
\end{obs}

\begin{teo}[propiedades de la densidad conjunta]
Una funci'on de densidad conjunta satisface
\begin{enumerate}
\item $\forall x,y \fxy(x,y) \geq 0$
\item $\iint_{\R^2} \fxy(x,y)\ dx\ dy = 
	\int_{-\infty}^\infty \int_{-\infty}^\infty \fxy(x,y)\ dx\ dy = 1$.
\end{enumerate}
\end{teo}

\begin{defi}[densidad marginal]
Las \emph{funciones de densidad marginal} de $X$ e $Y$ est'an dadas por:
$$f_X(x) = \int_{-\infty}^\infty \fxy(x,y)\ dy \mbox{\ \ y\ \ } 
	f_Y(y) = \int_{-\infty}^\infty \fxy(x,y)\ dx.$$
\end{defi}

\begin{defi}[acumulada conjunta continua] La \emph{funci'on de distribuci'on
acumulada conjunta} de $(X,Y)$ est'a dada por
$$\Fxy(x,y) = \int_{-\infty}^x \int_{-\infty}^y \fxy(s,t)\ dt\ ds.$$
\end{defi}

\begin{defi}[probabilidad condicional] Sea $(X,Y)$ un vector aleatorio. La 
funci'on de probabilidad condicional de $X$ dado que $Y=y$ est'a dada por
$$p_{X|Y=y}(x) = \frac{\pxy(x,y)}{p_Y(y)}.$$
\end{defi}

\begin{defi}[densidad condicional] Sea $(X,Y)$ un vector aleatorio. La 
funci'on de densidad condicional de $X$ dado que $Y=y$ est'a dada por
$$f_{X|Y=y}(x) = \frac{\fxy(x,y)}{f_Y(y)}.$$
\end{defi}

\begin{defi}[independencia]
$X$ e $Y$ son independientes si $\pxy(x,y)=p_X(x)p_Y(y)$
\end{defi}

\begin{obs}[independencia] 
Si $X$ e $Y$ son independientes $p_X = p_{X|Y=y}$ para todo $y$.
\end{obs}

\begin{defi}[esperanza de una funci'on de 2 v.a.]
$$E(h(X,Y)) = \sum_x \sum_y h(x,y) \pxy(x,y),$$
$$E(h(X,Y)) = \iint_{\R^2} h(x,y) \fxy(x,y)\ dx\ dy.$$
\end{defi}

\begin{teo}[linealidad de la esperanza]
$$E(aX + bY + c) = aE(X) + bE(Y) + c.$$
\end{teo}

\begin{teo}[esperanza de independientes]
Si $X$ e $Y$ son independientes entonces
$$E(XY) = E(X)E(Y).$$
\end{teo}

\begin{defi}[covarianza]
La covarianza de las v.a. $X$ e $Y$ se define como 
$\Cov(X,Y) = E[(X-\mu_X)(Y-\mu_Y)]$.
\end{defi}
La idea intuitiva es que la covarianza es cercana a $0$ cuanto mas
independientes son las variables. Si es lejana al $0$, el signo indica si la
correlaci'on es positiva o negativa.

\begin{obs}[covarianza y varianza]
$\Cov(X,X) = V(X)$.
\end{obs}

\begin{teo}[covarianza]
$$\Cov(X,Y) = E(XY) - E(X)E(Y).$$
\end{teo}

\begin{teo}[covarianza e independencia]
Si $X$ e $Y$ son independientes, entonces $\Cov(X,Y) = 0.$ La rec'iproca no
vale.
\end{teo}

\begin{defi}[coeficiente de relaci'on]
El \emph{coeficiente de relaci'on} es una estandarizaci'on de la covarianza
para que no dependa de las unidades. Se define como
$$\rho(X,Y) = \frac{\Cov(X,Y)}{\sigma_X \sigma_Y} =
	\frac{\Cov(X,Y)}{\sqrt{V(X)V(Y)}}.$$
\end{defi}

\begin{teo}[coeficiente de relaci'on]
El coeficiente de relaci'on cumple lo siguiente:
\begin{enumerate}
\item $\rho(aX+b,cY+d) = \frac{ab}{|ab|} \rho(X,Y).$
\item $-1 \leq \rho(X,Y) \leq 1$
\item $|\rho(X,Y)| = 1 \Leftrightarrow	Y = aX+b$ con probabilidad $1$. Notar
que el coeficiente mide la relaci'on lineal entre las variables.
\end{enumerate}
\end{teo}

\subsection{Extensi\'on a m\'as de dos dimensiones}

\begin{defi}[probabilidad conjunta]
Si $X_1,...,X_k$ son variables aleatorias discretas, su \emph{funci'on de
probabilidad conjunta} es 
$$p_{X_1,...,X_k}(x_1,...,x_k) = P(X_1 = x_1,...,X_k = x_k).$$
Dado $A \in \R^k$ 
$$P((X_1,...,X_k) \in A) = 
	\sum_{(x_1,...,x_k) \in A} p_{X_1,...,X_k}(x_1,...,x_k).$$
\end{defi}

\begin{teo}[propiedades de la probabilidad conjunta]
La probabilidad conjunta cumple:
\begin{enumerate}
\item $p_{X_1,...,X_k}(x_1,...,x_k) \geq 0$
\item $\sum_{(x_1,...,x_k) \in \R^k} p_{X_1,...,X_k}(x_1,...,x_k) = 1$
\end{enumerate}
\end{teo}

\begin{defi}[probabilidad marginal]
La \emph{probabilidad marginal} de un subvector $X_{i_1},...,X_{i_{k'}}$
est'a dada por:
$$p_{X_{i_1},...,X_{i_{k'}}}(x_{i_1},...,x_{i_{k'}}) = 
	\sum_{(x_{j_1},...,x_{j_{k-k'}}) \in \R^{k-k'}} 
		p_{X_1,...,X_k}(x_1,...,x_k),$$
donde $\{i_1,...,i_k,j_1,...,j_{k-k'}\} = \{1,...,k\}$.
\end{defi}

\subsection{Distribuci'on Multinomial}

Es una generalizaci'on de la distribuci'on binomial. Hay $n$ experimentos
posibles de $k$ resultados posibles con probabilidades $p_1,...,p_k$ tal que
$\sum_i p_i = 1$. Es vector aleatorio de la multinomial es $(X_1,...,X_k)$
donde $X_i$ es el n'umero de veces que sali'o el resultado $i$. Escribimos 
$$X \sim M(n,p_1,...,p_k).$$

\begin{defi}[probabilidad conjunta distribuci'on multinomial]
Si $(X_1,...,X_k) \sim M(n,p_1,...,p_k)$ entonces
$$p_{X_1,...,X_k}(x_1,...,x_k) = n! \prod_{i=1}^k \frac{p_i^{x_i}}{x_i!}.$$
\end{defi}

\begin{obs}
Si $(X_1,...,X_k) \sim M(n,p_1,...,p_l)$ entonces $X_i \sim Bi(n,p_i)$. En
general las marginales de una multinomial son binomiales o multinomiales.
\end{obs}

\begin{defi}[densidad conjunta]
$X_1,...,X_k$ son v.a. continuas si tienen una \emph{funci'on de
densidad conjunta} 
$$f_{X_1,...,X_k}(x_1,...,x_k).$$
Tal que dado $A \subseteq \R^k$ 
$$P((X_1,...,X_k) \in A) = 
	\int_A...\int p_{X_1,...,X_k}(x_1,...,x_k) dx_1...dx_k.$$
\end{defi}

\begin{teo}[propiedades de la densidad conjunta]
La densidad conjunta cumple:
\begin{enumerate}
\item $f_{X_1,...,X_k}(x_1,...,x_k) \geq 0$
\item $\int_{\R^k}...\int f_{X_1,...,X_k}(x_1,...,x_k) dx_1...dx_k = 
	\int_{-\infty}^\infty...\int_{-\infty}^\infty f_{X_1,...,X_k}(x_1,...,x_k) 
		dx_1...dx_k = 1$
\end{enumerate}
\end{teo}

\begin{defi}[densidad marginal]
La \emph{densidad marginal} de un subvector $X_{i_1},...,X_{i_{k'}}$ est'a dada
por:
$$f_{X_{i_1},...,X_{i_{k'}}}(x_{i_1},...,x_{i_{k'}}) = 
	\int_{(x_{j_1},...,x_{j_{k-k'}}) \in \R^{k-k'}}...\int
		f_{X_1,...,X_k}(x_1,...,x_k) dx_{j_1}...dx_{j_{k-k'}},$$
donde $\{i_1,...,i_k,j_1,...,j_{k-k'}\} = \{1,...,k\}$.
\end{defi}

\begin{defi}[independencia]
$X_1,...X_k$ son vectores aleatorios independientes sii:
$$p_{X_1,...,X_k}(x_1,...,x_k) = \prod_{i=1}^k p_{X_i}(x_i)\mbox{, o}$$
$$f_{X_1,...,X_k}(x_1,...,x_k) = \prod_{i=1}^k f_{X_i}(x_i).$$
\end{defi}

Para buscar la distribuci'on de una funci'on $g$ de varias variables aleatorias
de las que se sabe la conjunta, si es continua se puede plantear la ecuaci'on
$$P(g(X_1,...,X_k) = x) = 
	\sum_{(x_1,...,x_k) \in \R^k | g(x_1,...,x_k) = x}
		P((X_1,...,X_k) = (x_1,...,x_k)).$$
Para las continuas, por otro lado, se debe plantear
$$F_{g(X_1,...,X_k)}(x) = P(g(X_1,...,X_k) \leq x) = 
	\int_{(x_1,...,x_k) \in \R^k | g(x_1,...,x_k) \leq x} ... \int
		f_{X_1,...,X_k}(x_1,...,x_k) dx_1...dx_k.$$

\begin{teo}[Suma de binomiales]
Si $X \sim Bi(n,p)$ e $Y \sim Bi(m,p)$ son independientes, entonces 
$X+Y \sim Bi(n+m,p)$.
\end{teo}

\begin{teo}[Suma de poisson]
Si $X \sim P(\lambda)$ e $Y \sim P(\mu)$ son independientes, entonces 
$X+Y \sim P(\lambda + \mu)$.
\end{teo}

\begin{teo}[Suma de geom'etricas]
Si $X_i \sim G(p)$ son independientes, entonces $\sum_i X_i \sim BN(n,p)$.
\end{teo}

\begin{teo}[Suma de exponenciales]
Si $X \sim \expon(\lambda)$ e $Y \sim \expon(\lambda)$ son independientes, 
entonces $X+Y \sim \Gamma(2, \lambda)$.
\end{teo}

\begin{teo}[Suma de gamma]
Si $X \sim \Gamma(\alpha, \lambda)$ e $Y \sim \Gamma(\beta, \lambda)$ son
independientes, entonces $X+Y \sim \Gamma(\alpha+\beta, \lambda)$.
\end{teo}

\begin{teo}[Suma de normales]
Si $X \sim N(\mu,\sigma^2)$ e $Y \sim N(\mu', {\sigma^2}')$ son independientes,
entonces $aX+bY \sim N(a\mu+b\mu', a^2\sigma^2+b^2{\sigma^2}')$.
\end{teo}

\begin{teo}[Generadora de momentos de la suma]
Si $X$ e $Y$ son independientes entonces $$M_{X+Y}(t) = M_X(t) M_Y(t).$$
\end{teo}

\begin{teo}[Esperanza y varianza de la suma]
$$E(aX+bY) = aE(x)+bE(y) \mbox{\ \ y\ \ } 
	V(aX+bY) = a^2V(X)+b^2V(Y)+2ab\Cov(X,Y).$$
\end{teo}

\begin{teo}[Esperanza y varianza del promedio]
Sean $X_1,...,X_n$ v.a.i.i.d. (variables aleatorias independientes 
id'enticamente distribuidas) con $E(X_i) = \mu$ y $V(X_i) = \sigma^2$. 
Entonces 
$$E(\bar{X}) = \mu \mbox{\ \ y\ \ } V(\bar{X}) = \frac{\sigma^2}{n}.$$
\end{teo}

\begin{teo}[desigualdad de Chebyshev]
Si $X$ es una v.a. con $E(X) = \mu$ y $V(X) = \sigma^2$ entonces:
\begin{enumerate}
\item $\forall \eps > 0\ P(|X - \mu| > \eps) \leq \frac{\sigma^2}{\eps^2}$
\item $\forall k > 0\ P(|X - \mu| > k\sigma) \leq \frac{1}{k}$
\end{enumerate}
\end{teo}

\begin{defi}[l'imite en probabilidad]
Diremos que una sucesi'on de v.a. $X_n$ tiende en probabilidad a la v.a. $X$
y notaremos $X_i \tiendep X$ sii
$$\forall \eps > 0\ \limite{n}{\infty} P(|X_n - X| > \eps) = 0.$$
\end{defi}

\begin{teo}[ley de los grandes n'umeros]
Sean $X_1,X_2,...$ v.a.i.i.d. (muestra aleatoria) con $E(X_i) = \mu$ y 
$V(X_i) = \sigma^2 < \infty$, entonces $\bar{X_n} \tiendep \mu$ donde
$\bar{X_n} = \sum_{i=1}^n X_i / n$.
\end{teo}
Usando esta idea con la desigualdad de Chebyshev se puede encontrar la m'inima 
cantidad de repeticiones de un experimento para estimar la media con error
menor a un valor dado.

\begin{teo}[teorema central del l'imite]
Sean $X_1,X_2,...$ v.a.i.i.d. con $E(X_i)=\mu$ y $V(X_i)=\sigma^2 < \infty$,
entonces si $n$ es suficientemente grande
$$\frac{\sqrt(n)(\bar{X} - \mu)}{\sigma} \tiended Z \sim N(0,1),$$
donde $\tiended$ quiere decir que
$$P\left(\frac{\sqrt(n)(\bar{X} - \mu)}{\sigma} \leq a\right)\simeq\Phi(a).$$
\end{teo}

\begin{obs}[correcci'on por continuidad]
Cuando se usa el teorema central del l'imite para aproximar variables discretas
tiene un problema. Si se calcula $P(X \leq k) + P(X \geq k+1)$ deber'ia dar 1,
pero en la aproximaci'on se pierde la medida del intervalo $[k,k+1]$ que por
ser la aproximaci'on continua, no es 0. Para esto se estima $P(X \leq k)$ como
$P(X \leq (k+k')/2)$ donde $k'$ es el valor del rango de la v.a. inmediato 
siguiente a $k$.
\end{obs}

\begin{obs}[aproximaciones normales]
Algunas distribuciones, como la binomial, la poisson o la gamma, son sumas de 
s'i misma. Por lo tanto si el parametro sobre el cual se suma es
suficientemente grande, es posible aproximarla por una normal como si se
estuviera aproximando la suma.
\end{obs}

\section{Medidas de res'umen}

\begin{defi}[promedio o media muestral]
El \emph{promedio} o \emph{media muestral} de una muestra aleatoria 
$x_1,...,x_n$ est'a dado por
$$\bar{x} = \frac{1}{n}\sum_{i=1}^n x_i.$$
\end{defi}
La media muestra es muy sensible a la presencia de anomal'ias.

\begin{defi}[mediana muestral]
La \emph{mediana muestral} de una muestra aleatoria $x_1 \leq x_2 \leq ...
\leq x_n$ est'a dada por
$$\tilde{x} = 
	\begin{cases}
	x_k & x = 2k \\
	\frac{1}{2}(x_k+x_{k+1}) & x = 2k+1.
	\end{cases}$$
\end{defi}

\begin{defi}[media $\alpha$-podada]
La \emph{media $\alpha$-podada} de una muestra aleatoria $x_1 \leq x_2 \leq ...
\leq x_n$ se define como la media de $x_{[\alpha n] + 1},...,x_{n-[\alpha n]}$
\end{defi}

\begin{obs}[media, mediana y media $\alpha$-podada]
La media $\alpha$-podada es un intermedio entre media y mediana. La media es
la media $0$-podada y la mediana la media $(0.5-\eps)$-podada.
\end{obs}

\begin{defi}[rango muestral]
El \emph{rango muestral} de una muestra aleatoria $x_1 \leq x_2 \leq ...
\leq x_n$ se define como $x_n - x_1$.
\end{defi}
El rango muestral es muy sensible a la presencia de outliers.


\begin{defi}[varianza y desv'io muestral]
La \emph{varianza muestral} de una muestra aleatoria $x_1,...,x_n$ se define
como 
$$S^2 = \frac{\sum_{i=1}^n (x_i-\bar{x})^2}{n-1}$$
y el desv'io como $S = \sqrt{S^2}$.
\end{defi}

\begin{defi}[coeficiente de variaci'on]
El \emph{coeficiente de variaci'on} de una muestra aleatoria $x_1,...,x_n$
son desv'io $S$ se define como $S / \bar{x}$.
\end{defi}

\begin{defi}[percentiles y cuartiles]
El \emph{percentil} $100 \alpha$ de la muestra $x_1,...,x_n$ es 
$x_{\lceil\alpha (n+1)\rceil}$ si $\alpha (n+1)$ es entero o mayor a $n$ y 
$(x_{\lfloor\alpha (n+1)\rfloor}+x_{\lceil\alpha (n+1)\rceil})/2$ en otro caso.
Llamaremos primer, segundo y tercer \emph{cuartil} a los percentiles $25$, 
$50$ y $75$ respectivamente. El primer y tercer cuartiles tambien se llaman
cuartil inferior y superior.
\end{defi}

\begin{defi}[distancia intercuartil]
Se define como la diferencia entre los cuartiles inferior y superior.
\end{defi}
Es menos sensible a outliers que la varianza, el desv'io y el rango.

\begin{defi}[cuartos y distancia intercuartos]
El \emph{cuarto inferior} es la mediana de la mitad inferior de la muestra
(redondeado hacia arriba). El \emph{cuarto superior} es la mediana de la mitad
superior. La \emph{distancia intercuartos} es la diferencia entre ambos.
\end{defi}
Cuartos y cuartiles son muy similares.

\begin{defi}[desv'io absoluto mediano]
El \emph{desv'io absoluto mediano} de la muestra $x_1 \leq x_2 \leq ... \leq 
x_n$ se define como la mediana de las distancias a la mediana
$\MAD = \mbox{mediana}(|x_i - \tilde{x}|)$.
\end{defi}

\begin{defi}[n'umeros de res'umen]
Los 5 n'umeros de res'umen de una muestra son el m'inimo, el m'aximo, la
mediana y los cuartiles.
\end{defi}

\section{Gr\'aficos}


\textbf{Tallo hoja}. Se pone la parte mas significativa de la medici'on como
tallo y se usa para agrupar. Como hoja se pone la parte menos significativa
(se repite una sola vez el tallo por cada grupito y luego una lista de las
hojas).


\textbf{Tallo hoja espalda con espalda}. Para comparar 2 grupos de mediciones,
se usa el mismo tallo y se ponen ambas listas de hojas una hacia cada lado.


\textbf{Histograma}. Se divide la muestra en intervalos disjuntos y se grafica
una barra para cada intervalo con area igual a la frecuencia (o frecuencia 
relativa) de las mediciones dentro del intervalo.


\textbf{Box-plot}. Se hace una caja entre los dos cuartiles con un segmento en
el medio indicando la mediana. Luego se hacen rectas que salen de la caja hasta
el valor mas alejado hacia cada lado menor o igual a 1.5 de la distancia
intercuartil (bigotes). Los valores que quedan fuera de este rango se
representan con puntos individuales y son considerados an'omalos (outliers).


\textbf{Box-plot} paralelos. Para comparar varios conjuntos de muestras, se 
hace un box plot para cada una paralelos con la misma escala en el eje Y.


\textbf{QQ-plot}. Sirve para ver si una muestra se parece una distribuci'on. 
En el eje X se ponen los puntos observados ordenados y en el eje Y percentiles 
equidistribuidos de la distribuci'on te'orica y se observa cu'anto se parece a
una recta el gr'afico obtenido.

\section{M\'etodos de estimaci\'on puntual}

Nota: Notaremos con $X_1,...,X_n$ las v.a. que representan la medici'on antes
de hacerla y como $x_1,...,x_n$ los valores de las mediciones propiamente
dichas. Cualquier funci'on de los $X_i$ sera una v.a.

\begin{defi}[momentos muestrales]
Dada una muestra aleatoria (m.a.) $X_1,...,X_n$ definimos el momento muestral 
de 'orden $k$ como 
$$\frac{1}{n} \sum_{i=1}^m X_i^k.$$
\end{defi}

\begin{defi}[estimaci'on de momentos]
La \emph{estimaci'on de momentos} de los $k$
par'ametros de una distribuci'on dada una muestra $X_1,...,X_n$ estima
los par'ametros como las soluciones del sistema de ecuaciones que resulta
de igualar los primeros momentos de la distribuci'on con el correspondiente
momento muestral (la cantidad necesaria de momentos a igualar dependen de la
distribuci'on).
\end{defi}

\begin{teo}[estimador de momentos de la exponencial]
Si $X_1,...,X_n$ es una m.a. de una distribuci'on $\expon(\lambda)$ entonces
su estimador de momentos es $\hat{\lambda} = 1 / \bar{X}$.
\end{teo}

\begin{teo}[estimador de momentos de la gamma]
Si $X_1,...,X_n$ es una m.a. de una distribuci'on $\Gamma(\alpha,  \lambda)$
entonces su estimador de momentos es 
$$\hat{\lambda} = \frac{\bar{X}}{-\bar{X}^2 + \frac{1}{n}\sum_{i=1}^n X_i^2}
\mbox{\ \ y\ \ }
\hat{\alpha} = \frac{\bar{X}^2}{-\bar{X}^2 + \frac{1}{n}\sum_{i=1}^n X_i^2}.$$
\end{teo}

\begin{teo}[estimador de momentos de la uniforme]
Si $X_1,...,X_n$ es una m.a. de una distribuci'on $U(0,\theta)$ entonces
su estimador de momentos es $\hat{\theta} = 2\bar{X}$.
\end{teo}

\begin{defi}[funci'on de verosimilitud y estimaci'on de m'axima verosimilitud]
Sea una distribuci'on con funci'on de probabilidad puntual $p_{\vec{X}}$ o de
densidad $f_{\vec{X}}$ que depende de par'ametros $\theta_1,...,\theta_k$. Si
$x_1,...,x_n$ son los valores observados la \emph{funci'on de verosimilitd}
$L(\theta_1,...,\theta_k)$ se define como $p$ o $f$ en el punto $x_1,...,x_n$
usando $\theta_1,...,\theta_k$ como par'ametros. La \emph{estimaci'on de
m'axima verosimilitud (EMV)} es el punto m'aximo de $L$.
\end{defi}

\begin{teo}[estimador de m'axima verosimilitud de la binomial]
Sea $X_1,...,X_n$ una m.a. de un experimento binomial de par'ametro $p$.
Su EMV es $\hat{p} = \bar{X}$.
\end{teo}

\begin{teo}[estimador de m'axima verosimilitud de la exponencial]
Sea $X_1,...,X_n$ una m.a. de una distribuci'on $\expon(\lambda)$, entonces
su EMV es $\hat{\lambda} = 1 / \bar{X}$.
\end{teo}

\begin{teo}[estimador de m'axima verosimilitud de la normal]
Sea $X_1,...,X_n$ una m.a. de una distribuci'on $N(\mu,\sigma^2)$, entonces
su EMV es 
$$\hat{\mu} = \bar{X} \mbox{\ \ y \ \ } 
	\hat{\sigma^2} = \frac{1}{n} \sum_{i=1}^n (X_i - \bar{X})^2.$$
\end{teo}

\begin{teo}[estimador de m'axima verosimilitud de la uniforme]
Sea $X_1,...,X_n$ una m.a. de una distribuci'on $U(0,\theta)$, entonces
su EMV es $\hat{\theta} = \max_i(X_i)$.
\end{teo}

\begin{teo}[invarianza de los EMV]
Si $\tilde{\theta}_1,...,\tilde{\theta}_n$ son estimadores de m'axima
verosimilitud de $\theta_1,...,\theta_n$ entonces
para toda funci'on inyectiva $f$ $f(\tilde{\theta}_1,...,\tilde{\theta}_n)$ 
lo es de $f(\theta_1,....,\theta_n)$.
\end{teo}

\begin{defi}[sesgo]
El \emph{sesgo} de un estimador $\tilde{\theta}$ del par'ametro $\theta$ se
define como $b(\tilde{\theta}) = E_\theta(\tilde{\theta}) - \theta$. Un
estimador $\tilde{\theta}$ se dice \emph{insesgado} si $b(\tilde{\theta})=0$.
\end{defi}

\begin{defi}[asint'oticamente insesgado]
Un estimador $\tilde{\theta}$ es \emph{asint'oticamente insesgado} si
$\limite{n}{\infty} E_\theta(\tilde{\theta}) = \theta$.
\end{defi}

\begin{teo}[sesgo del estimador de momentos de la uniforme]
El estimador de momentos $\tilde{\theta}$ de una $U(0,\theta)$ es insesgado.
\end{teo}

\begin{teo}[sesgo de EMVs conocidos]
El EMV $\tilde{p}$ de una $Bi(n,p)$ es insesgado. El EMV $\tilde{\mu}$ de una
normal es insesgado, pero el EMV $\tilde{\sigma^2}$ s'olo es asint'oticamente
insesgado. El EMV $\tilde{\theta}$ de una $U(0,\theta)$ s'olo es 
asint'oticamente insesgado.
\end{teo}

\begin{teo}[la varianza muestral es insesgado de la varianza]
La varianza muestral $S^2$ es un estimador insesgado de $\sigma^2$ para
cualquier distribuci'on.
\end{teo}

\begin{defi}[estimador insesgado de m'inima varianza]
Se define como el \emph{estimador insesgado de varianza m'inima} (IMVU) al
estimador $\tilde{\theta}$ insesgado tal que para todo estimador insesgado
$\tilde{\theta}'$ pasa que 
$V_\theta(\tilde{\theta}) \leq V_\theta(\tilde{\theta}')$.
\end{defi}

\begin{teo}[estimador IMVU de la normal]
Si la m.a. $X_1,...,X_n$ viene de una $N(\mu,\sigma^2)$ entonces $\bar{X}$ es
el estimador IMVU de $\mu$.
\end{teo}

\begin{defi}[error est'andar y error est'andar estimado]
Si $\tilde{\theta}$ es un estimador su \emph{error est'andar} es 
$\sqrt{V_\theta(\tilde{\theta})}$. Reemplazando en dicha expresi'on los
par'ametros desconocidos por estimadores de los mismos se define el
\emph{error est'andar estimado}.
\end{defi}

\begin{teo}[error de estimaci'on de la normal]
Si la m.a. $X_1,...,X_n$ viene de una $N(\mu,\sigma^2)$ y consideramos el
estimador $\bar{X}$ de $\mu$ su error est'andar es $\sqrt{\sigma^2/n}$ y
estimado $\sqrt{S^2/n}$.
\end{teo}

\begin{defi}[error cuadr'atico medio]
El \emph{error cuadr'atico medio} de $\tilde{\theta}$
es $\ECM_\theta(\tilde{\theta}) = E_\theta[(\tilde{\theta}-\theta)^2]$.
\end{defi}

\begin{teo}[relaci'on entre ECM y varianza]
$\ECM_\theta(\tilde{\theta}) = V_\theta(\tilde{\theta}) + b(\tilde{\theta})^2$.
\end{teo}

\begin{defi}[estimador de menor ECM]
Se define un estimador como tomar, entre varios estimadores, el de menor
ECM. Es una forma de hacer trade-off entre la varianza y el sesgo. Para
estimadores insesgados, el criterio es obviamente igual al de m'inima varianza.
\end{defi}

\begin{defi}[consistencia]
Un estimador $\tilde{\theta}_n$ basado en la muestra $X_1,...,X_n$ es
\emph{consistente} si $\tilde{\theta}_n \tiendep \theta$.
\end{defi}

\begin{teo}[media consistente]
$\bar{X}$ es un estimador consistente de $\mu = E(X)$.
\end{teo}

\begin{teo}[varianza consistente]
$S^2_X$ es un estimador consistente de $\sigma^2 = V(X)$.
\end{teo}

\begin{teo}[consistencia]
Si 
$$\limite{n}{\infty} E_\theta(\tilde{\theta}_n) = \theta \mbox{\ \ y \ \ }
\limite{n}{\infty} V_\theta(\tilde{\theta}_n) = 0$$
el estimador $\tilde{\theta}_n$ consistente.
\end{teo}

\section{Intervalos de confianza}

\begin{defi}[intervalo de confianza]
Sea $X_1,...,X_n$ una m.a. dependiente de un par'ametro $\theta$ y dos
funciones de la misma a y b tal que
$$P(a(X_1,...,X_n) \leq \theta \leq b(X_1,...,X_n)) = 1-\alpha,$$
con $\alpha$ cercano a $0$. Se llama \emph{intervalo de confianza} para 
$\theta$ de nivel $1-\alpha$ a $[a(X_1,...,X_n),b(X_1,...,X_n)]$.
\end{defi}

\begin{defi}[distribuci'on de Student]
Si $Z \sim N(0,1)$ y $U \sim \chi^2_n = \Gamma(n/2,1/2)$ son independientes
entonces
$$T = \frac{Z}{\sqrt(U/n)} \sim t_n$$
donde $t_n$ es la \emph{distribuci'on de student} con $n$ grados de libertad.
Est'a tabulada para distintos valores de $n$ y tiende a una normal standard
cuando $n \rightarrow \infty$.
\end{defi}

\begin{teo}
Sea $X_1,...,X_n$ una m.a. de una $N(\mu,\sigma^2)$, se cumple que:
\begin{enumerate}
\item $\bar{X} \sim N(\mu,\sigma^2/n) \Leftrightarrow 
	\sqrt{n}\frac{\bar{X}-\mu}{\sigma} \sim N(0,1)$
\item $(n-1)S^2 / \sigma^2 \sim \chi^2_{n-1}$
\item $\bar{X}$ y $S^2$ son independientes.
\item $\sqrt{n}\frac{\bar{x}-\mu}{S} \sim t_{n-1}$
\end{enumerate}
\end{teo}

\textbf{Notaci'on}: $z_\alpha = P(X \sim N(0,1) \leq \alpha)$, 
$t_{n,\alpha} = P(X \sim Student(n) \geq \alpha)$, 
$\chi^2_{n,\alpha} = P(X \sim \chi^2_n \geq \alpha)$.

\begin{teo}[intervalos de confianza para la normal]
Si $X_1,...,X_n$ es una m.a. de distribucion $N(\mu,\sigma^2)$ y queremos
intervalos de confianza de nivel $1-\alpha$ entonces
\begin{enumerate}
\item Si $\sigma^2 = \sigma^2_0$ es conocida, el intervalo para $\mu$ es
$$\left[\bar{X}-\frac{z_{1-\alpha/2}\sigma_0}{\sqrt{n}},
	\bar{X}+\frac{z_{1-\alpha/2}\sigma_0}{\sqrt{n}}\right].$$
\item Si $\mu = \mu_0$ es conocida, el intervalo para $\sigma^2$ es
$$\left[\frac{\displaystyle \sum_{i=1}^n (X_i - \mu_0)^2}
			 {\chi^2_{n,1-\alpha/2}},
	\frac{\displaystyle \sum_{i=1}^n (X_i - \mu_0)^2}
		 {\chi^2_{n,\alpha/2}}\right].$$
\item Si son ambos par'ametros desconocidos el intervalo para $\mu$ es
$$\left[\bar{X}-t_{n-1,\alpha/2} \frac{S}{\sqrt{n}},
	\bar{X}+t_{n-1,\alpha/2} \frac{S}{\sqrt{n}}\right]$$
y para $\sigma$
$$\left[\frac{(n-1)S^2}{\chi^2_{n-1,1-\alpha/2}},
	\frac{(n-1)S^2}{\chi^2_{n-1,\alpha/2}}\right].$$
\end{enumerate}
\end{teo}

\begin{defi}[pivote]
Sea $X_1,...,X_n$ una m.a. dependiente de $\theta$, se llama \emph{pivote} a
una funci'on $T(X_1,...,X_n,\theta)$ cuya distribuci'on no depende de $\theta$
ni de ning'un otro par'ametro. 
\end{defi}
Entonces, se toman $a$ y $b$ tal que
$$P(a \leq T(X_1,...,X_n,\theta) \leq b) = 1 - \alpha.$$
Despejando se obtiene un intervalo de confianza para $\theta$.

\begin{defi}[intervalos de confianza de nivel asint'otico]
Sea $X_1,X_2,...$ una m.a. dependiente de un par'ametro $\theta$ y dos
sucesiones de funciones de la misma a y b tal que
$$\limite{n}{\infty} P(a_n(X_1,...,X_n) \leq \theta \leq P(b_n(X_1,...,X_n))
	= 1-\alpha,$$
con $\alpha$ cercano a $0$. Se llama \emph{sucesi'on de intervalos de
confianza} para $\theta$ de \emph{nivel asint'otico} $1-\alpha$ a 
$[a_n(X_1,...,X_n),b_n(X_1,...,X_n)]$. Tambi'en se dice que para $n$
suficientemente grande $[a_n(X_1,...,X_n),b_n(X_1,...,X_n)]$ tiene nivel
aproximado $1-\alpha$.
\end{defi}

\begin{teo}[intervalos de confianza aproximados por la normal]
Como $\sqrt{n}(\bar{X}-\mu)/\sigma \tiended N(0,1)$ podemos usar la normal
para determinar intervalos de confianza aproximados. En particular el intervalo
de confianza de nivel aproximado $1-\alpha$ de $\mu = E(X_i)$ con $\sigma$
desconocido es
$$\left[\bar{X}-z_{\alpha/2}\frac{s}{\sqrt{n}},
	\bar{X}+z_{\alpha/2}\frac{s}{\sqrt{n}}\right].$$
\end{teo}

\section{Test de hip'otesis}

\begin{defi}[hip'otesis]
Se denomina \emph{hip'otesis nula} y se nota $H_0$ a la hip'otesis que implica
el statu quo. Por otro lado, la \emph{hip'otesis alternativa} o 
\emph{hip'otesis del investigador} $H_1$ es la que implica un cambio.
\end{defi}

\begin{defi}[zona de rechazo]
La \emph{zona de rechazo} es el conjunto de valores para los cuales
se rechaza la hip'otesis nula. Su forma depende de como sea la hip'otesis
alternativa.
\end{defi}

\begin{defi}[test de hip'otesis]
Un \emph{test de hip'otesis} se basa en un estad'istico (funci'on de la
muestra) y una zona de rechazo.
\end{defi}

\begin{defi}[p-valor]
Definimos como p-valor a la probabilidad de que el estad'istico tenga un valor
menor o igual al observado si $H_0$ es cierta (suponiendo que $H_1$ es de la
forma $\theta > \theta_0$ y que el estad'istico es creciente en $\theta$).
\end{defi}

\begin{defi}[nivel de significaci'on del test]
El \emph{nivel de significaci'on del test}, notado $\alpha$ es la probabilidad
de de rechazar $H_0$ si es cierta. A la probabilidad de no rechazarla dado que
no es cierta se la nota por $\beta$.
\end{defi}
En general construimos el estad'istico suponiendo que $H_0$ es cierta, asi que
podemos controlar $\alpha$ y es lo que haremos.

\begin{defi}[funci'on de potencia]
La \emph{funci'on de potencia} de un test $\pi(\mu)$ es la 
probabilidad de rechazar $H_0$ cuando el verdadero valor del par'ametro es
$\mu$ (i.e., la probabilidad de la zona de rechazo). Depende de ambos tipos de 
error, pues
$$\pi(\mu) =
	\begin{cases}
		\alpha(\mu) & \mu \in H_0 \\
		1-\beta(\mu) & \mu \in H_1 \\
	\end{cases}$$
\end{defi}

\begin{teo}[test para la normal de varianza conocida]
Para testear contra el valor $\mu_0$ el estad'istico del test es
$$T = \sqrt{n} \frac{\bar{X}-\mu_0}{\sigma_0} \sim N(0,1)$$
Seg'un cual sea $H_1$ la zona de rechazo y la funci'on de potencia ser'an
\begin{enumerate} 
\item $H_1 = \mu > \mu_0$: $T \geq z_\alpha$, 
$\displaystyle \pi(\mu) = 
	1-\Phi\left(z_\alpha+\sqrt{n}\frac{\mu_0-\mu}{\sigma_0}\right)$.
\item $H_1 = \mu < \mu_0$: $T \leq -z_\alpha$,
$\displaystyle \pi(\mu) = 
	\Phi\left(-z_\alpha+\sqrt{n}\frac{\mu_0-\mu}{\sigma_0}\right)$.
\item $H_1 = \mu \neq \mu_0$: $|T| \geq z_{\alpha/2}$, 
$\displaystyle \pi(\mu) = 
	1-\Phi\left(z_{\alpha/2}+\sqrt{n}\frac{\mu_0-\mu}{\sigma_0}\right)
	 +\Phi\left(-z_{\alpha/2}+\sqrt{n}\frac{\mu_0-\mu}{\sigma_0}\right)$.
\end{enumerate}
\end{teo}

\begin{teo}[test para la normal de varianza desconocida]
Para testear contra el valor $\mu_0$ el estad'istico del test es
$$T = \sqrt{n} \frac{\bar{X}-\mu_0}{S} \sim t_{n-1}$$
Seg'un cual sea $H_1$ la zona de rechazo ser'a
\begin{enumerate} 
\item $H_1 = \mu > \mu_0$: $T \geq t_{n-1,\alpha}$.
\item $H_1 = \mu < \mu_0$: $T \leq -t_{n-1,\alpha}$.
\item $H_1 = \mu \neq \mu_0$: $|T| \geq t_{n-1,\alpha/2}$.
\end{enumerate}
\end{teo}

\begin{teo}[test para la varianza de la normal con media desconocida]
Para testear contra el valor $\mu_0$ el estad'istico del test es
$$T = \frac{(n-1)S^2}{\sigma_0^2} \sim \chi^2_{n-1}$$
Seg'un cual sea $H_1$ la zona de rechazo ser'a
\begin{enumerate} 
\item $H_1 = \mu > \mu_0$: $T \geq \chi^2_{n-1,\alpha}$.
\item $H_1 = \mu < \mu_0$: $T \leq \chi^2_{n-1,1-\alpha}$.
\item $H_1 = \mu \neq \mu_0$: $T \geq \chi^2_{n-1,\alpha/2}$ 'o 
	$T \leq \chi^2_{n-1,1-\alpha/2}$.
\end{enumerate}
\end{teo}

\begin{teo}[test aproximado para la media de una distribuci'on cualquier]
Como $X_n \tiended N(0,1)$ podemos usar un test similar al de la normal usando
$s$ en lugar de $\sigma$ (ya que $s \rightarrow \sigma$).
\end{teo}

\newpage

\section{Res'umen de distribuciones}
\label{sec:resumen}

\subsection{Distribuciones Discretas}
$$
\begin{array}{l|l|c|c|c|c}
\mathbf{\mbox{Nombre}} & \mathbf{X \sim} & \mathbf{p_X(k)}
	& \mathbf{E(X)} & \mathbf{V(X)} &  \mathbf{M_X(t)} \\\hline\hline
\mbox{Binomial} & Bi(n,p) & \displaystyle {n \choose k}p^k(1-p)^{n-k} & 
	np & np(1-p) & \displaystyle (e^t p + 1 - p)^n \\\hline
\mbox{Geom'etrica} & G(p) & (1-p)^{k-1}p & \bigfrac{1}{p} & \bigfrac{1-p}{p^2}
	& \bigfrac{pe^t}{1-(1-p)e^t}\\\hline
\mbox{Binomial Negativa} & BN(p,r) & 
	\displaystyle {k-1 \choose r-1} (1-p)^{k-r}p^r &
	\bigfrac{r}{p} & \bigfrac{r(1-p)}{p^2} & 
	\left(\bigfrac{p e^t}{1 - (1-p)e^t}\right)^r \\\hline
\mbox{Hipergeom'etrica} & H(n,N,D) & 
	\frac{ \displaystyle {D \bigchoose k} {N-D \bigchoose n-k} }
		 { \displaystyle {N \bigchoose n}  } & n\bigfrac{D}{N} &
	\left(\bigfrac{N-n}{N-1}\right)n\bigfrac{D}{N}\left(1-\bigfrac{D}{N}\right)
	\\\hline
\mbox{Poisson} & P(\lambda) & \bigfrac{e^{-\lambda} \lambda^k}{k!} & 
	\lambda & \lambda & e^{\displaystyle \lambda (e^t - 1)}\\\hline
\end{array}
$$

\subsection{Distribuciones Continuas}

$$
\begin{array}{l|l|c|c|c|c|c}
\mathbf{\mbox{Nombre}} & \mathbf{X \sim} & \mathbf{f_X(x)} & \mathbf{F_X(x)} &
	\mathbf{E(X)} & \mathbf{V(X)} & \mathbf{M_X(t)} \\\hline\hline
\mbox{Uniforme} & U(A,B) & \bigfrac{1}{B-A} I_{[A,B]}(x) & 
	\bigfrac{x-A}{B-A} I_{[A,B]}(x) & \bigfrac{A+B}{2} & 
	\bigfrac{(B-A)^2}{12} & \bigfrac{e^{tb} - e^{ta}}{t(b - a)} \\\hline
\mbox{Normal} & N(\mu,\sigma^2) & \bigfrac{1}{\sqrt{2 \pi} \sigma} 
	e ^ {\displaystyle -\frac{(x-\mu)^2}{2 \sigma^2}} & \Phi(x)
	& \mu & \sigma^2 & e^{\bigfrac{\sigma^2 t^2}{2}+\mu t }\\\hline
\mbox{Gamma} & \Gamma(\alpha,\lambda) & 
	\bigfrac{e^{-\lambda x} x^{\alpha-1} \lambda^\alpha}{\Gamma(\alpha)} &
	& \bigfrac{\alpha}{\lambda} & \bigfrac{\alpha}{\lambda^2} 
	& \left(\bigfrac{\lambda}{\lambda - t}\right)^\alpha \\\hline
\mbox{Exponencial} & \expon(\lambda) & 
	e^{-\lambda x} \lambda & 1-e^{-\lambda x} &
	\lambda^{-1} & \lambda^{-2} & \bigfrac{\lambda}{\lambda - t} \\\hline
\end{array}
$$

\subsection{Distribuci\'on Vectorial Discreta}
$$
\begin{array}{l|l|c}
\mathbf{\mbox{Nombre}} & \mathbf{X \sim} & 
	\mathbf{p_{X_1,...,X_k}(x_1,...,x_k)} \\\hline\hline
\mbox{Multinomial} & M(n,p_1,...,p_k) & 
	\displaystyle n! \prod_{i=1}^k \frac{p_i^{x_i}}{x_i!} \\\hline
\end{array}
$$


\label{theend}
\end{document}
